% !TeX root = ../main.tex

% \section{Introduction}

Understanding the movement and behaviour of animals is crucial for their conservation, and one common statistical model used to do so is the hidden Markov model, or HMM \citep{Sutherland:1998, Ogburn:2017, McClintock:2020}. An HMM is a generalization of a mixture model that is used to decode a latent process of interest (e.g., a sequence of animal behaviours) from an observed time series (e.g., biologging data from tags attached to the animal). They have been used to uncover a wide variety of animal behaviours, including foraging activity \citep{Lusseau:2009, Ylitalo:2023} and habitat selection \citep{Klappstein:2023}.

Many ecological studies employ \textit{unsupervised} HMMs, meaning that the true behaviours of the study animals are never directly observed and instead are predicted entirely from biologging data \citep{Barajas:2017, Patterson:2017, Pirotta:2018, Adam:2019}. However, ecologists are often interested in predicting complicated animal behaviours (e.g., successful prey captures, Tennessen et al., 2019b)\nocite{Tennessen:2019b} that are difficult to identify from movement data alone. For these behaviours, the relationship between an animal’s behaviour and its movement is so complex that it is rarely fully characterized by a statistical model.  

One solution is to fully observe an animal's behaviours and incorporate them into the underlying model, in which case the HMM is \textit{fully supervised}. \citet{Krogh:1997} show that fully supervised HMMs exhibit better predictive performance than unsupervised HMMs in various settings, and fully supervised HMMs are used in fields ranging from speech recognition to medicine \citep{Bagos:2003, Tamposis:2018}. To my knowledge fully supervised HMMs are rare in ecology, but some animal behaviour studies use other fully supervised machine learning techniques \citep{Carroll:2014, Allen:2016}. However, these studies often focus on captive animals that are much easier to continuously observe compared to wild animals.

While fully observing an animal's behaviour in the wild can be prohibitively difficult or expensive, many ecological studies have behavioural information for a small subset of time. Occasional observations of an animal's behaviour can be incorporated into a \textit{semi-supervised} HMM, and some notable ecological studies have used semi-supervised HMMs. For example, \citet{McClintock:2012} label a subset of hidden behavioural states of a grey seal (\textit{Halichoerus grypus}) using its proximity to known ``haul-out" and foraging sites. \citet{Pirotta:2018} assume that northern fulmars (\textit{Fulmarus glacialis}) begin every journey in some known behavioural state. \citet{McRae:2024} use drone footage to directly label the behaviour of killer whales (\textit{Orcinus orca}) for a subset of observation times. \citet{Saldanha:2023} use a multi-sensor approach to derive behavioural labels for red-billed tropicbirds (\textit{Phaethon aethereus}). These studies demonstrate that incorporating partial labels into an ecological HMM can significantly improve its performance.

While behavioural labels can improve an HMM's prediction accuracy, many ecological studies only have access to labels for a small proportion of observations \citep[e.g., $< 10$\%,][]{Saldanha:2023, McRae:2024}. In these cases, the labelled data often do not meaningfully affect the parameter estimates of an HMM because the likelihood is dominated by the unlabelled data \citep{Chapelle:2006, Ren:2020}. A study by \citet{Ji:2009} uses a weighted likelihood approach to increase the influence of labelled examples, but it assumes that labels correspond to independent time series. This approach does not apply when labels occur \textit{within} a time series, which is often the case for ecological studies \citep{Saldanha:2023, McRae:2024}. 

Foraging behaviour can be especially rare and difficult to identify, but it is often of prime interest in ecology \citep{Stephens:2008, Saldanha:2023}. For example, understanding foraging behaviour is vital for the conservation of northern and southern resident killer whales off the coast of British Columbia \citep{Lusseau:2009, Joy:2019, Tennessen:2023}. The two sub-populations overlap in their spatial distribution and dietary preferences, but the population of the threatened northern residents has been steadily growing for decades while the population of the endangered southern residents has remained stagnant \citep{DFO:2018,Tennessen:2023}.  %Both northern and southern resident killer whales feed on Chinook salmon (\textit{Oncorhynchus tshawytscha}) that reside in deeper waters and in few numbers compared to other types of salmon \citep{Ford:2006, Ford:2009}. Both sub-populations must therefore expend a considerable amount of energy when they forage \citep{Williams:2009, Noren:2011, Wright:2017}. 
Studies have shown that various factors contribute to these population trends, including prey availability, anthropogenic pollutants, and vessel disturbances, but the exact causal mechanisms are not fully understood \citep{Lusseau:2009, Joy:2019, Murray:2021}. Each of these factors affects foraging behaviour and foraging success differently, so understanding how often and how successfully these sub-populations hunt may help explain differences in their growth \citep{Noren:2011, Tennessen:2023}. 
 
In this chapter, I introduce a novel weighted semi-supervised learning approach for hidden Markov models that allows practitioners to adjust the influence of sparse labels within a time series. 
%It naturally incorporates labels within the time series itself, and it does not rely on imputing hidden labels to perform inference.
Section \ref{sec:chp3_background} reviews the definition of an HMM and current semi-supervised learning techniques for %non-HMM
mixture models that lack time dependence. Section \ref{sec:chp3_mod} formalizes a \textit{partially hidden Markov model}, or PHMM, which is designed to account for time-series that are partially labelled, and introduces a weighted likelihood approach to balance the influence of labelled and unlabelled data within the model. Section \ref{sec:chp3_case} presents two case studies that use the weighted likelihood to acheive higher cross-validated accuracy compared to existing baseline methods. Section \ref{sec:chp3_discussion} discusses my results.

% It is much harder to collect labels of foraging activity in free-swimming killer whales \citep{Tennessen:2019a}, so killer whale kinematic data is an ideal use case to incorporate semi-supervised learning within HMMs.

%Thanks to recent advances in tracking technology, ecologists are now collecting an unprecedented amount of movement data, developing more complicated HMMs, and identifying increasingly complex behaviours \citep{Glennie:2023}. 

% the foraging activity of toothed whales (odontocetes) can be affected by whale-watching and fishing vessels \citep{Hamer:2012, Senigaglia:2016} or by man-made sound affecting their ability to echolocate \citep{Weilgart:2007}. It is thus important to understand the foraging behaviour of odontocetes to predict how anthropogenic factors can affect their ability to search for and capture prey \citep{Tennessen:2019a}.

% while directly observing animals can be difficult, researchers can use statistical models to infer unobserved behaviours from biologging data. Thanks to recent advances in tracking technology, ecologists are now collecting an unprecedented amount of movement and kinematic data for a wide variety of animals \citep{Patterson:2017,Nathan:2022}. Such data allow them to characterize behaviours such as hunting \citep{Heerah:2017} and habitat selection \citep{Michelot:2019b}.

%can be used to predict the unlabelled behaviours within the process of interest or those within a separate and fully unlabelled test data set 

%Toothed whales (odontocetes) can be affected by whale-watching and fishing vessels \citep{Hamer:2012, Senigaglia:2016} or by man-made sound affecting their ability to echolocate \citep{Weilgart:2007}. It is thus important to understand the foraging behaviour of odontocetes to predict how anthropogenic factors can affect their ability to search for and capture prey \citep{Tennessen:2019a}. 

%Thus, my goal is to identify foraging behaviour and predict successful prey capture events.