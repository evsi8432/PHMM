% !TeX root = ../main.tex

% \section{Introduction}

%\textcolor{red}{Note that this chapter will go after the first chapter on HMMs for functional data. As such, some introduction on HMMs has been omitted. Perhaps it is best to move to put in the first chapter sooner rather than later.}

Recent advances in tracking technology allow ecologists to collect an unprecedented amount of movement and kinematic data for a wide variety of animals \citep{Patterson:2017}. These data sets allow researchers to detect hunting behaviour \citep{Heerah:2017}, understand habitat selection \citep{Michelot:2019b}, and develop activity budgets \citep{Dot:2016} for these animals. Understanding animals' movement and behaviour can assist in their conservation \citep{Sutherland:1998} \citep{Ogburn:2017}.

Animal movement data is collected on various time scales and with varying regularity. For example, some animal position data is sampled irregularly at an average rate of less than one observation per hour \citep{Gryba:2019}. At the same time, some accelerometers collect data regularly at a rate of 50 observations per second \citep{Daneault:2021}. Partially in response to this wide variety and plethora of data, statistical models for animal movement and behaviour have become more complicated in recent years \citep{Hooten:2017}. These models have therefore become more challenging to perform parameter inference over \citep{McClintock:2012, Michelot:2019b}.

One of the most prevalent models used to describe animal movement is the hidden Markov model, or HMM \citep{McClintock:2020}. Although they are effective on relatively coarse scales, standard HMMs rely on several unrealistic assumptions for fine-scale processes. For example, traditional HMMs assume that the process dictating an animal’s behaviour does not change over time. This assumption is violated if some coarse-scale process (e.g. dive type for a marine animal) affects the animal's fine-scale behaviour (e.g. searching for prey vs chasing prey). In addition, HMMs model observations as independent of one another when conditioned on the animal's underlying behaviour. However, fine-scale processes usually exhibit intricate dependence structures with a high degree of autocorrelation (e.g., the acceleration of an animal at a given time is highly correlated with its acceleration 20 milliseconds later). One additional concern with many HMMs that researchers do not know which behavioural states the HMM will discover \textit{a priori}. Instead, they must interpret an animal's behavioural states after the HMM estimates them.

While high-frequency data sets can reveal fine-scale animal behaviours, many location data sets are sparse and irregularly sampled in time. In particular, marine animal location is often recorded using satellites \citep{Gryba:2019} or by researchers visually observing the animal when it surfaces \citep{Hartman:2020}. Hidden Markov models fail when observations are not equispaced in time, so many ecological statisticians instead implement continuous-time methods based on diffusion processes \citep{Blackwell:2016, Michelot:2019b}. Movement models based on diffusion processes can handle irregular time intervals, but they are often seen as more challenging to fit and less interpretable than discrete-time models.

Parameter estimation for both HMMs and continuous-time methods for complex animal behaviours is computationally expensive. For example, the hierarchical hidden Markov model (HHMM) from the case study of chapter 2 requires trial-and-error to perform model selection and takes hours to fit. In addition, sparsely sampled diffusion processes often have intractable likelihoods. Researchers can approximate the likelihood by inferring an animal's position between observations, but inferring unobserved positions is often computationally expensive \citep{Lindstrom:2012}.

%Relatively recent advances in biologging technology have resulted in an explosion of increasingly high-frequency and large-scale data sets detailing the location and movement of humans as well as a wide variety of animals \cite{Patterson:2017}. These rich data sets allow researchers to describe animal behaviour in extremely fine detail. However, these data sets also pose a challenge for statisticians and biologists since increasingly complicated models are required to sufficiently describe the fine-scale processes reflected within. 

The examples above highlight the importance of developing accurate statistical models and efficient inference methods for animal movement at various scales. As such, the overall goals of this thesis are: (1) to implement statistical models that accurately describe animal movement and behaviour using data sets on a variety of scales, and (2) to develop novel inference procedures to mitigate the computational burden of implementing these statistical models.

The primary case study I will use in this thesis involves modelling the movement and behaviour of northern and southern resident killer whales (\textit{Orcinus orca}) off the coast of British Columbia, Canada. The northern resident population is slowly growing and is designated as threatened, while the southern resident population is classified as endangered \citep{DFO:2018}. Fitting accurate statistical models to this data set will help researchers understand each killer whale's energy expenditure as well as discrepancies between the two sub-populations \citep{Green:2009, Dot:2016, Wilson:2019}. Understanding energy expenditure may also aid in conservation efforts for the northern and southern residents \citep{Noren:2011}.

An HMM can be applied in one of three settings: \textit{unsupervised, supervised,} or \textit{semi-supervised}. In an \textit{unsupervised} setting, the hidden state process $\{X_t\}_{t=1}^T$ is completely unobserved. This setting is the most common situation in ecological modelling \citep{Adam:2019, Pirotta:2018, Barajas:2017, Patterson:2017}. In a \textit{supervised} setting, a training data set includes a completely observed hidden state process $\{X_t\}_{t=1}^T$ and is used for parameter estimation \citep{Krough:1994}. Supervised HMMs are often trained with the goal of accurately predicting the hidden state process of some separate test data set. Supervised HMMs have been shown to exhibit better predictive performance compared to HMMs without labels \citep{Krogh:1997}. Supervised HMMs are commonly used in biomedical applications \citep{Tamposis:2018} and speech recognition \citep{Bagos:2003}, and have been for decades. Finally, HMMs can be used in a \textit{semi-supervised} setting, in which the hidden state process is only \textit{partially} observed \citep{Tamposis:2018,Bagos:2003,Li:2021}. Within semi-supervised learning, data sets range from \textit{densely labelled}, where most hidden states are observed, to \textit{sparsely labelled}, where most hidden states are unobserved. Semi-supervised techniques are useful when collecting labels for every observation is expensive or time-consuming \citep{Scheffer:2001}. Such situations are common in ecological studies, where animals are often only occasionally directly observed. 

Due to the difficulty in obtaining labelled data, ecological studies tend to use unsupervised methods \citep{Patterson:2017,McClintock:2020}, with some notable semi-supervised exceptions. For example, \citet{McClintock:2012} labels a subset of hidden behavioural states for a grey seal (\textit{Halichoerus grypus}) using its proximity to known ``haul-out" and foraging sites. In addition, \citet{Pirotta:2018} assumes that northern fulmars (\textit{Fulmarus glacialis}) begin every journey in some known behavioural state. 

Both of the studies listed above use the \textit{MomentuHMM} package in R, which can incorporate labels into hidden Markov models \citep{McClintock:2018}. In order to accoutn for labels, the package slightly alters the traditional HMM likelihood from Equation (\ref{eqn:HMM_like}) and maximizes the result. In particular, if $x_t$ is known for some $t \in \{1,\ldots,T\}$, then \citet{McClintock:2018} set all elements of the emission distribution matrix $P(y_t;\theta)$ to zero except for the lone entry corresponding to the labelled hidden state. This modified likelihood corresponds to a probability model in which label observation times are non-random.

To maximize the altered HMM likelihood, \citet{McClintock:2018} use gradient-based methods, while \citet{Li:2021} independently develop a modified Baum-Welch algorithm to maximize the same objective function. \citet{Tamposis:2018} introduce an iterative method similar to the traditional EM algorithm in which they repeatedly estimate the HMM parameters and impute the hidden state process. %While intuitive, the modified Baum-Welch algorithm does not work well in sparsely-labelled settings since sequences of labelled data are required to obtain the original initial parameter estimates in this algorithm. In addition, it is not clear why the traditional Baum-Welch algorithm would not be preferable to this approach.

All of the approaches above implicitly assume that researchers know in advance when hidden states will be observed. As a result, they fail to explicitly model the presence or absence of a label itself into the HMM, which can be useful for parameter inference and state decoding. If the probability of observing a hidden state $i \in \{1,\ldots,N\}$ at time $t \in \{1,\ldots,T\}$ depends upon either the observation ($Y_t$) or the hidden state ($X_t$), then it is important to model these dependencies into the HMM to avoid model misspecification and bias in parameter estimation.
%
%\citet{Ramasso:2014} use a method based on belief functions to incorporate prior knowledge of the hidden states. However, this method relies on Dempster-Shafer theory and requires practitioners to define labels using counter-intuitive belief functions which do not necessarily have the same properties as traditional probability measures.
%

One application of interest here is the identification of foraging behaviour in killer whales off the coast of British Columbia. I am interested in modelling sequences of killer whale dives with an HMM and explicitly labelling foraging as a hidden behavioural state. Subject matter experts have used video evidence to confirm seven prey capture events among dive profiles of eleven individual killer whales. Foraging was the only categorized behaviour, so the probability of observing the killer whale's behavioural state depended upon the behavioural state itself. In addition, video evidence was not conclusive in all cases, so it is unclear if the probability of observing a label (i.e. conclusive video evidence) depended on the observation itself. For example, more ``extreme" foraging behaviour may have been easier to label as foraging behaviour. To my knowledge, this issue is not widely addressed in current semi-supervised HMM or ecological statistics literature. 

Another limitation with the formulation of \citet{McClintock:2018} occurs if the data are sparsely labelled and the HMM is misspecified. In this situation, the likelihood from Equation (\ref{eqn:HMM_like}) may be dominated by the unlabelled data such that labelled data do not meaningfully affect maximum likelihood parameter estimates. The issue of label sparsity and model misspecification is well known in machine learning \citep{Chapelle:2006,Ren:2020}, but it is not widely addressed in applied HMM and statistical ecology literature, especially since semi-supervised models are relatively rare to begin with.
 
This chapter aims to introduce and review semi-supervised HMMs and apply them to the field of ecology, where labels tend to be sparse or unused. I will explore the difficulties caused by sparse labels and adapt solutions from the machine learning literature into an ecological statistics framework. As a novel contribution, I will model labels as observations from an HMM to capture the dependence between the hidden state process, observation process, and label observation process.

This chapter is organized as follows: Section \ref{sec:prob_obs_ss} formalizes a \textit{partially-observed} HMM (PHMM) and shows how models from previous studies can be expressed as PHMMs. Section \ref{sec:prob_obs_ss} also contains a simulation study that demonstrates issues that arise when a misspecified model is fit to data generated from a PHMM. Section \ref{sec:w_like_ss} reviews a weighted likelihood approach to balance the influence of labelled and unlabelled data in a PHMM. Finally, Section \ref{sec:fut_ss} proposes simulation studies, case studies, and evaluation criteria that will test the effectiveness of the PHMM and weighted likelihood approach.